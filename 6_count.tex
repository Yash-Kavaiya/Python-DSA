\documentclass[aspectratio=169]{beamer}
\usepackage[utf8]{inputenc}
\usepackage{xcolor}
\usepackage{hyperref}
\usepackage{tikz}
\usepackage{amsmath}
\usepackage{listings}

% Google Color Palette
\definecolor{GoogleBlue}{RGB}{66,133,244}
\definecolor{GoogleRed}{RGB}{219,68,55}
\definecolor{GoogleYellow}{RGB}{244,160,0}
\definecolor{GoogleGreen}{RGB}{15,157,88}
\usetheme{Madrid}
\setbeamercolor{frametitle}{bg=GoogleBlue, fg=white}
\setbeamercolor{title}{bg=GoogleBlue,fg=white}

% Custom header/footer as per expert theme
\setbeamertemplate{headline}{
  \leavevmode%
  \hbox{%
    \begin{beamercolorbox}[wd=.5\paperwidth,ht=2.5ex,dp=1ex,left]{palette quaternary}%
      \hspace*{2ex}\textcolor{white}{Python DSA: Count Digits}
    \end{beamercolorbox}%
    \begin{beamercolorbox}[wd=.5\paperwidth,ht=2.5ex,dp=1ex,right]{palette quaternary}%
      \textcolor{white}{Slide \insertframenumber}\hspace*{2ex}
    \end{beamercolorbox}}%
  \vskip0pt%
}
\setbeamertemplate{footline}{
  \leavevmode%
  \hbox{%
    \begin{beamercolorbox}[wd=.33\paperwidth,ht=2.5ex,dp=1ex,left]{palette primary}%
      \hspace*{2ex}\href{https://easy-ai-labs.lovable.app/}{\textcolor{white}{Easy AI Labs}}
    \end{beamercolorbox}%
    \begin{beamercolorbox}[wd=.34\paperwidth,ht=2.5ex,dp=1ex,center]{palette secondary}%
      \href{https://www.linkedin.com/in/yashkavaiya}{\textcolor{white}{Yash Kavaiya}}
    \end{beamercolorbox}%
    \begin{beamercolorbox}[wd=.33\paperwidth,ht=2.5ex,dp=1ex,right]{palette tertiary}%
      \href{https://www.linkedin.com/company/genai-guru}{\textcolor{white}{Gen AI Guru}}\hspace*{2ex}
    \end{beamercolorbox}}%
  \vskip0pt%
}

\title{\textbf{Count Digits in a Number}}
\subtitle{DSA Approaches: While Loop, Logarithmic}
\author{Presented by Expert}
\date{\today}

\begin{document}

% Title
\begin{frame}
\titlepage
\end{frame}

% Concept
\begin{frame}{Count Digits - Problem Statement}
\textbf{Input:} An integer $n$

\textbf{Output:} Total number of digits in $n$

\textbf{Examples:}
\begin{itemize}
  \item $5438 \rightarrow 4$
  \item $177715 \rightarrow 6$
  \item $973 \rightarrow 3$
\end{itemize}
\end{frame}

% Visual step-by-step (while approach)
\begin{frame}{Iterative Solution - Visual Steps}
\begin{block}{Algorithm}
\begin{itemize}
  \item Set $count=0$, $num=n$
  \item While $num>0$: $count+=1$ and $num \gets num // 10$
  \item Return $count$
\end{itemize}
\end{block}
\begin{center}
\begin{tikzpicture}[node distance=1.6cm, scale=0.85]
\node[draw, fill=GoogleYellow!10, rounded corners] (n1) {num=5438, count=0};
\node[draw, fill=GoogleRed!10,rounded corners,below=of n1] (n2) {num=543, count=1};
\node[draw, fill=GoogleGreen!10,rounded corners,below=of n2] (n3) {num=54, count=2};
\node[draw, fill=GoogleBlue!10,rounded corners,below=of n3] (n4) {num=5, count=3};
\node[draw, rounded corners,below=of n4] (n5) {num=0, count=4};
\draw[->,thick] (n1) -- (n2);
\draw[->,thick] (n2) -- (n3);
\draw[->,thick] (n3) -- (n4);
\draw[->,thick] (n4) -- (n5);
\end{tikzpicture}
\end{center}
\end{frame}

% Python code iterative
\begin{frame}[fragile]{Python Code - Iterative While}
\begin{lstlisting}[language=Python]
def count_digits(n):
    num = n
    count = 0
    while num > 0:
        count += 1
        num = num // 10
    return count

print(count_digits(5438))    # Output: 4
print(count_digits(177715))  # Output: 6
print(count_digits(973))     # Output: 3
\end{lstlisting}
\end{frame}

% Logarithmic shortcut
\begin{frame}{Logarithmic Shortcut}
\textbf{Formula:}
\[
\text{digits} = \lfloor \log_{10}(n) + 1 \rfloor
\]

\textbf{Examples:}
\begin{itemize}
  \item $\log_{10}(5438) = 3.735 \rightarrow 4$
  \item $\log_{10}(177715) = 5.249 \rightarrow 6$
  \item $\log_{10}(973) = 2.988 \rightarrow 3$
\end{itemize}
\end{frame}

% Python code logarithmic
\begin{frame}[fragile]{Python Code - Logarithmic Approach}
\begin{lstlisting}[language=Python]
from math import log10

def count_digits(num):
    return int(log10(num) + 1)

print(count_digits(5438))    # Output: 4
print(count_digits(177715))  # Output: 6
print(count_digits(973))     # Output: 3
\end{lstlisting}
\end{frame}

% Complexity slide
\begin{frame}{Time and Space Complexity}
\textbf{Iterative Approach:}
\begin{itemize}
  \item Time: $O(\log_{10} n)$ (proportional to number of digits)
  \item Space: $O(1)$
\end{itemize}
\textbf{Logarithmic Formula:}
\begin{itemize}
  \item Time: $O(1)$ (one math function call)
  \item Space: $O(1)$
\end{itemize}
\end{frame}

% Footer/Connect slide
{
\setbeamertemplate{headline}{}
\begin{frame}[fragile]{Content Details \& Follow for Updates}
\begin{block}{Presentation Content}
\begin{itemize}
\item \textbf{Topic:} Python Count Digits Two-Approach
\item \textbf{Includes:} Visual step, code, formulas, complexity discussion
\end{itemize}
\end{block}

\begin{block}{Connect With Us}
\centering
\begin{tabular}{ll}
\textcolor{GoogleBlue}{\textbf{Website:}} & \href{https://easy-ai-labs.lovable.app/}{easy-ai-labs.lovable.app} \\
\textcolor{GoogleRed}{\textbf{LinkedIn:}} & \href{https://www.linkedin.com/in/yashkavaiya}{Yash Kavaiya} \\
\textcolor{GoogleGreen}{\textbf{Company:}} & \href{https://www.linkedin.com/company/genai-guru}{Gen AI Guru} \\
\textcolor{GoogleYellow!80!black}{\textbf{YouTube:}} & \href{https://www.youtube.com/@genai-guru}{@genai-guru} \\
\end{tabular}
\end{block}

\begin{center}
\colorbox{GoogleBlue!10}{\begin{minipage}{11cm}
\centering
\textbf{Follow for more: DSA | Python | Coding Interview Tricks}
\end{minipage}}
\end{center}
\end{frame}
}

\end{document}
