\documentclass[aspectratio=169]{beamer}
\usepackage[utf8]{inputenc}
\usepackage{tikz}
\usepackage{amsmath}
\usepackage{xcolor}
\usepackage{hyperref}
\usepackage{graphicx}
\usepackage{booktabs}

% Google Colors
\definecolor{GoogleBlue}{RGB}{66,133,244}
\definecolor{GoogleRed}{RGB}{219,68,55}
\definecolor{GoogleYellow}{RGB}{244,160,0}
\definecolor{GoogleGreen}{RGB}{15,157,88}

% Theme
\usetheme{Madrid}
\usecolortheme{default}
\setbeamercolor{frametitle}{bg=GoogleBlue, fg=white}
\setbeamercolor{title}{bg=GoogleBlue,fg=white}
\setbeamercolor{block title}{bg=GoogleGreen,fg=white}
\setbeamercolor{block body}{bg=GoogleGreen!10,fg=black}

% Custom Header/Footer
\setbeamertemplate{headline}{
  \leavevmode%
  \hbox{%
  \begin{beamercolorbox}[wd=.5\paperwidth,ht=2.5ex,dp=1ex,left]{palette quaternary}%
    \hspace*{2ex}\textcolor{white}{Python DSA Operations}
  \end{beamercolorbox}%
  \begin{beamercolorbox}[wd=.5\paperwidth,ht=2.5ex,dp=1ex,right]{palette quaternary}%
    \textcolor{white}{Slide \insertframenumber}\hspace*{2ex}
  \end{beamercolorbox}}%
  \vskip0pt%
}
\setbeamertemplate{footline}{
  \leavevmode%
  \hbox{%
  \begin{beamercolorbox}[wd=.33\paperwidth,ht=2.5ex,dp=1ex,left]{palette primary}%
    \hspace*{2ex}\href{https://easy-ai-labs.lovable.app/}{\textcolor{white}{Easy AI Labs}}
  \end{beamercolorbox}%
  \begin{beamercolorbox}[wd=.34\paperwidth,ht=2.5ex,dp=1ex,center]{palette secondary}%
    \href{https://www.linkedin.com/in/yashkavaiya}{\textcolor{white}{Yash Kavaiya}}
  \end{beamercolorbox}%
  \begin{beamercolorbox}[wd=.33\paperwidth,ht=2.5ex,dp=1ex,right]{palette tertiary}%
    \href{https://www.linkedin.com/company/genai-guru}{\textcolor{white}{Gen AI Guru}}\hspace*{2ex}
  \end{beamercolorbox}}%
  \vskip0pt%
}

% Title Details
\title{\textbf{Python List, Set, Dictionary Operation Time Complexities}}
\subtitle{DSA Interview Preparation}
\author{Presented by Expert}
\institute{Google Colors DSA Series}
\date{\today}

\begin{document}

% Title Slide
{
\setbeamertemplate{headline}{}
\setbeamertemplate{footline}{}
\begin{frame}
\titlepage
\vspace{-0.8cm}
\begin{center}
\begin{tikzpicture}
\node[fill=GoogleBlue!12, rounded corners, minimum width=12cm, minimum height=2cm] {
\begin{minipage}{11cm}
\centering
\textcolor{GoogleBlue}{\textbf{Coding Interviews में फेल ना हों!}}
\vskip 0.10cm
\textcolor{GoogleRed}{हर List, Set, Dictionary Operation की}
\vskip 0.02cm
\textcolor{GoogleGreen}{\footnotesize Time Complexity जानते हैं} 
\end{minipage}
};
\end{tikzpicture}
\end{center}
\end{frame}
}

% Agenda
\begin{frame}{Agenda}
\begin{block}{इस Presentation में:}
\begin{itemize}
  \item Python List के Common Operations \& Time Complexity
  \item Set, Dictionary के Average/Worst Case Complexity
  \item DSA Interviews में क्यों ज़रूरी हैं ये Details
  \item टेबल/चार्ट्स के साथ Comparison और Examples
\end{itemize}
\end{block}
\end{frame}

% Importance
\begin{frame}{क्यों ज़रूरी है Complexity जानना?}
\textbf{हर लाइन के कोड के पीछे Time Complexity है!}
\vskip0.2cm
\begin{itemize}
  \item किसी भी Operation (append, pop, insert, delete) का टाइम कैसे बढ़ता है size के साथ?
  \item Interview में कोड/Logic के साथ सही Complexity बतानी होती है
  \item अगर सही Complexity नहीं समझी, तो TLE (Time Limit Exceeded) आ सकता है!
  \item Efficient कोड में सही Data Structure और Method Selection ज़रूरी है
\end{itemize}
\end{frame}

% Primary Table (from your image)
\begin{frame}{List Operations - Time Complexity Table}
\centering
\small
\begin{tabular}{lcc}
\rowcolor{GoogleBlue!10}
\textbf{Operation} & \textbf{Average Case} & \textbf{Amortized Worst Case} \\
\midrule
Copy                & $O(n)$    & $O(n)$ \\
Append              & $O(1)$    & $O(1)$ \\
Pop last            & $O(1)$    & $O(1)$ \\
Pop intermediate    & $O(n)$    & $O(n)$ \\
Insert              & $O(n)$    & $O(n)$ \\
Get Item            & $O(1)$    & $O(1)$ \\
Set Item            & $O(1)$    & $O(1)$ \\
Delete Item         & $O(n)$    & $O(n)$ \\
Iteration           & $O(n)$    & $O(n)$ \\
Get Slice           & $O(k)$    & $O(k)$ \\
Del Slice           & $O(n)$    & $O(n)$ \\
Set Slice           & $O(k+n)$  & $O(k+n)$ \\
Extend              & $O(k)$    & $O(k)$ \\
Sort                & $O(n\log n)$ & $O(n\log n)$ \\
Multiply            & $O(nk)$   & $O(nk)$ \\
\texttt{x in s}     & $O(n)$    & $O(n)$ \\
\texttt{min(s), max(s)} & $O(n)$ & $O(n)$ \\
\texttt{len(s)}     & $O(1)$    & $O(1)$ \\
\end{tabular}
\end{frame}

% Explanation Slide 1
\begin{frame}{List Main Operations Explanation}
\textcolor{GoogleRed}{\textbf{Append, Pop last, Get/Set Item:}} $O(1)$ (Constant Time)
\begin{itemize}
    \item List के अंत में Fast insert/delete
    \item Index से access/update भी Constant Time
\end{itemize}
\vskip0.2cm
\textcolor{GoogleBlue}{\textbf{Pop Intermediate, Insert, Delete Item:}} $O(n)$ (Linear Time)
\begin{itemize}
    \item List के बीच में डालना/निकालना मतलब सभी elements को शिफ्ट करना होता है
    \item इसलिए ज़्यादातर समय $n$ पर निर्भर करता है
\end{itemize}
\end{frame}

% Explanation Slide 2
\begin{frame}{List Other Important Ops}
\textcolor{GoogleGreen}{\textbf{Sort:}} $O(n\log n)$
\begin{itemize}
    \item Built-in sort, merge sort, quick sort सब $n\log n$ के आस-पास होते हैं
\end{itemize}
\textcolor{GoogleYellow}{\textbf{x in s, min(s), max(s):}} $O(n)$
\begin{itemize}
    \item किसी भी item के लिए पूरा list check होता है worst case में
\end{itemize}
\texttt{len(s):} $O(1)$
\begin{itemize}
    \item Length Python internally store करता है, instantly देता है
\end{itemize}
\end{frame}

% Set Complexity (Hindi explanation high-yield ops)
\begin{frame}{Set Operations - Complexity}
\textbf{Set Membership, Get, Set, Delete:}
\begin{itemize}
    \item Average Case: $O(1)$ (Constant Time, hash table)
    \item Worst/Amortized Case: $O(n)$ (hash collision scenario - rare)
    \item Interview में बोले हमेशा average $O(1)$, ब्लू मून में $O(n)$
\end{itemize}
\end{frame}

% Dictionary Complexity
\begin{frame}{Dictionary Operations - Complexity}
\textbf{Dictionary Get, Set, Delete:}
\begin{itemize}
    \item Average Case: $O(1)$ (Constant, hash lookup)
    \item Amortized Worst Case: $O(n)$ (rare hash collision)
    \item Copying entire dict: $O(n)$ (size के proportional)
\end{itemize}
\textbf{Note:} Interview में अगर पूछा जाए, always tell average $O(1)$, but acknowledge worst $O(n)$ in theory
\end{frame}

% Key Interview Table (Side-by-side)
\begin{frame}{List, Set, Dict - Quick Interview Table}
\centering
\small
\begin{tabular}{lccc}
\rowcolor{GoogleYellow!10}
\textbf{Operation} & \textbf{List} & \textbf{Set} & \textbf{Dict} \\
\midrule
Get item       & $O(1)$    & $O(1)$    & $O(1)$    \\
Set item       & $O(1)$    & $O(1)$    & $O(1)$    \\
Append/Insert  & $O(1)$/O(n) & $O(1)$  & $O(1)$    \\
Delete         & $O(n)$    & $O(1)$    & $O(1)$    \\
Search (in)    & $O(n)$    & $O(1)$    & $O(1)$    \\
Copy           & $O(n)$    & $O(n)$    & $O(n)$    \\
Sort           & $O(n\log n)$ & N/A     & N/A       \\
\end{tabular}
\end{frame}

% Practical Tips
\begin{frame}{प्रैक्टिकल इंटरव्यू टिप्स}
\begin{itemize}
    \item हमेशा बताओ: Constant vs Linear Time
    \item Update/Search में List vs Set/Dict: Set/Dict always $O(1)$ (on average)
    \item Rarely interview पूछेगा: worst-case collisions ($O(n)$)
    \item Built-in functions fastest हैं, direct ऑपरेशन अगर possible है
    \item लिस्ट के बीच में बार-बार insert/delete से बचें, वरना TLE आ सकता है
\end{itemize}
\end{frame}

% Takeaways
\begin{frame}{Key Takeaways}
\begin{itemize}
    \item हर Data Structure के Common Operations की Complexity याद करना जरूरी
    \item Efficient code और Interview दोनों के लिए High Confidence देता है
    \item Tables को Revise करते रहें, हर Leetcode/DSA सवाल में इनका प्रयोग करें
    \item Built-in sort, hash-based search python में super optimized है
\end{itemize}
\centering
\textcolor{GoogleBlue}{\textbf{Practice, Revise and Crack Coding Interviews!}}
\end{frame}

% Content/Connect Slide
{
\setbeamertemplate{headline}{}
\begin{frame}{Content Details \& Follow for More Updates}
\begin{block}{Presentation Content}
\begin{itemize}
\item \textbf{Topic:} Python Data Structure Operations - Time Complexity
\item \textbf{Coverage:} List, Set, Dict: Average/Worst Case, Interview Tables, Practical Tips
\item \textbf{Level:} Beginner to Pro: Interview & Leetcode Ready
\item \textbf{Target:} CS Students | DSA | Coding Interviews
\end{itemize}
\end{block}
\begin{block}{Connect With Us}
\centering
\begin{tabular}{ll}
\textcolor{GoogleBlue}{\textbf{Website:}} & \href{https://easy-ai-labs.lovable.app/}{easy-ai-labs.lovable.app} \\
\textcolor{GoogleRed}{\textbf{LinkedIn:}} & \href{https://www.linkedin.com/in/yashkavaiya}{Yash Kavaiya} \\
\textcolor{GoogleGreen}{\textbf{Company:}} & \href{https://www.linkedin.com/company/genai-guru}{Gen AI Guru} \\
\textcolor{GoogleYellow!80!black}{\textbf{YouTube:}} & \href{https://www.youtube.com/@genai-guru}{@genai-guru} \\
\end{tabular}
\end{block}
\begin{center}
\colorbox{GoogleBlue!15}{
\begin{minipage}{11cm}
\centering
\textbf{Follow for more updates: DSA | Coding Interview | Python Performance}
\end{minipage}
}
\end{center}
\end{frame}
}

\end{document}
